
%% bare_conf.tex
%% V1.4b
%% 2015/08/26
%% by Michael Shell
%% See:
%% http://www.michaelshell.org/
%% for current contact information.
%%
%% This is a skeleton file demonstrating the use of IEEEtran.cls
%% (requires IEEEtran.cls version 1.8b or later) with an IEEE
%% conference paper.
%%
%% Support sites:
%% http://www.michaelshell.org/tex/ieeetran/
%% http://www.ctan.org/pkg/ieeetran
%% and
%% http://www.ieee.org/

%%*************************************************************************
%% Legal Notice:
%% This code is offered as-is without any warranty either expressed or
%% implied; without even the implied warranty of MERCHANTABILITY or
%% FITNESS FOR A PARTICULAR PURPOSE! 
%% User assumes all risk.
%% In no event shall the IEEE or any contributor to this code be liable for
%% any damages or losses, including, but not limited to, incidental,
%% consequential, or any other damages, resulting from the use or misuse
%% of any information contained here.
%%
%% All comments are the opinions of their respective authors and are not
%% necessarily endorsed by the IEEE.
%%
%% This work is distributed under the LaTeX Project Public License (LPPL)
%% ( http://www.latex-project.org/ ) version 1.3, and may be freely used,
%% distributed and modified. A copy of the LPPL, version 1.3, is included
%% in the base LaTeX documentation of all distributions of LaTeX released
%% 2003/12/01 or later.
%% Retain all contribution notices and credits.
%% ** Modified files should be clearly indicated as such, including  **
%% ** renaming them and changing author support contact information. **
%%*************************************************************************


% *** Authors should verify (and, if needed, correct) their LaTeX system  ***
% *** with the testflow diagnostic prior to trusting their LaTeX platform ***
% *** with production work. The IEEE's font choices and paper sizes can   ***
% *** trigger bugs that do not appear when using other class files.       ***                          ***
% The testflow support page is at:
% http://www.michaelshell.org/tex/testflow/



\documentclass[conference]{IEEEtran}
% Some Computer Society conferences also require the compsoc mode option,
% but others use the standard conference format.
%
% If IEEEtran.cls has not been installed into the LaTeX system files,
% manually specify the path to it like:
% \documentclass[conference]{../sty/IEEEtran}





% Some very useful LaTeX packages include:
% (uncomment the ones you want to load)


% *** MISC UTILITY PACKAGES ***
%
%\usepackage{ifpdf}
% Heiko Oberdiek's ifpdf.sty is very useful if you need conditional
% compilation based on whether the output is pdf or dvi.
% usage:
% \ifpdf
%   % pdf code
% \else
%   % dvi code
% \fi
% The latest version of ifpdf.sty can be obtained from:
% http://www.ctan.org/pkg/ifpdf
% Also, note that IEEEtran.cls V1.7 and later provides a builtin
% \ifCLASSINFOpdf conditional that works the same way.
% When switching from latex to pdflatex and vice-versa, the compiler may
% have to be run twice to clear warning/error messages.






% *** CITATION PACKAGES ***
%
%\usepackage{cite}
% cite.sty was written by Donald Arseneau
% V1.6 and later of IEEEtran pre-defines the format of the cite.sty package
% \cite{} output to follow that of the IEEE. Loading the cite package will
% result in citation numbers being automatically sorted and properly
% "compressed/ranged". e.g., [1], [9], [2], [7], [5], [6] without using
% cite.sty will become [1], [2], [5]--[7], [9] using cite.sty. cite.sty's
% \cite will automatically add leading space, if needed. Use cite.sty's
% noadjust option (cite.sty V3.8 and later) if you want to turn this off
% such as if a citation ever needs to be enclosed in parenthesis.
% cite.sty is already installed on most LaTeX systems. Be sure and use
% version 5.0 (2009-03-20) and later if using hyperref.sty.
% The latest version can be obtained at:
% http://www.ctan.org/pkg/cite
% The documentation is contained in the cite.sty file itself.






% *** GRAPHICS RELATED PACKAGES ***
%
%\ifCLASSINFOpdf\
\usepackage{graphicx}
\usepackage{subfig}
\usepackage{amsmath}
  % declare the path(s) where your graphic files are
  % \graphicspath{{../pdf/}{../jpeg/}}
  % and their extensions so you won't have to specify these with
  % every instance of \includegraphics
  % \DeclareGraphicsExtensions{.pdf,.jpeg,.png}
%\else
  % or other class option (dvipsone, dvipdf, if not using dvips). graphicx
  % will default to the driver specified in the system graphics.cfg if no
  % driver is specified.
  % \usepackage[dvips]{graphicx}
  % declare the path(s) where your graphic files are
  % \graphicspath{{../eps/}}
  % and their extensions so you won't have to specify these with
  % every instance of \includegraphics
  % \DeclareGraphicsExtensions{.eps}
%\fi
% graphicx was written by David Carlisle and Sebastian Rahtz. It is
% required if you want graphics, photos, etc. graphicx.sty is already
% installed on most LaTeX systems. The latest version and documentation
% can be obtained at: 
% http://www.ctan.org/pkg/graphicx
% Another good source of documentation is "Using Imported Graphics in
% LaTeX2e" by Keith Reckdahl which can be found at:
% http://www.ctan.org/pkg/epslatex
%
% latex, and pdflatex in dvi mode, support graphics in encapsulated
% postscript (.eps) format. pdflatex in pdf mode supports graphics
% in .pdf, .jpeg, .png and .mps (metapost) formats. Users should ensure
% that all non-photo figures use a vector format (.eps, .pdf, .mps) and
% not a bitmapped formats (.jpeg, .png). The IEEE frowns on bitmapped formats
% which can result in "jaggedy"/blurry rendering of lines and letters as
% well as large increases in file sizes.
%
% You can find documentation about the pdfTeX application at:
% http://www.tug.org/applications/pdftex





% *** MATH PACKAGES ***
%
%\usepackage{amsmath}
% A popular package from the American Mathematical Society that provides
% many useful and powerful commands for dealing with mathematics.
%
% Note that the amsmath package sets \interdisplaylinepenalty to 10000
% thus preventing page breaks from occurring within multiline equations. Use:
%\interdisplaylinepenalty=2500
% after loading amsmath to restore such page breaks as IEEEtran.cls normally
% does. amsmath.sty is already installed on most LaTeX systems. The latest
% version and documentation can be obtained at:
% http://www.ctan.org/pkg/amsmath





% *** SPECIALIZED LIST PACKAGES ***
%
%\usepackage{algorithmic}
% algorithmic.sty was written by Peter Williams and Rogerio Brito.
% This package provides an algorithmic environment fo describing algorithms.
% You can use the algorithmic environment in-text or within a figure
% environment to provide for a floating algorithm. Do NOT use the algorithm
% floating environment provided by algorithm.sty (by the same authors) or
% algorithm2e.sty (by Christophe Fiorio) as the IEEE does not use dedicated
% algorithm float types and packages that provide these will not provide
% correct IEEE style captions. The latest version and documentation of
% algorithmic.sty can be obtained at:
% http://www.ctan.org/pkg/algorithms
% Also of interest may be the (relatively newer and more customizable)
% algorithmicx.sty package by Szasz Janos:
% http://www.ctan.org/pkg/algorithmicx




% *** ALIGNMENT PACKAGES ***
%
%\usepackage{array}
% Frank Mittelbach's and David Carlisle's array.sty patches and improves
% the standard LaTeX2e array and tabular environments to provide better
% appearance and additional user controls. As the default LaTeX2e table
% generation code is lacking to the point of almost being broken with
% respect to the quality of the end results, all users are strongly
% advised to use an enhanced (at the very least that provided by array.sty)
% set of table tools. array.sty is already installed on most systems. The
% latest version and documentation can be obtained at:
% http://www.ctan.org/pkg/array


% IEEEtran contains the IEEEeqnarray family of commands that can be used to
% generate multiline equations as well as matrices, tables, etc., of high
% quality.




% *** SUBFIGURE PACKAGES ***
%\ifCLASSOPTIONcompsoc
%  \usepackage[caption=false,font=normalsize,labelfont=sf,textfont=sf]{subfig}
%\else
%  \usepackage[caption=false,font=footnotesize]{subfig}
%\fi
% subfig.sty, written by Steven Douglas Cochran, is the modern replacement
% for subfigure.sty, the latter of which is no longer maintained and is
% incompatible with some LaTeX packages including fixltx2e. However,
% subfig.sty requires and automatically loads Axel Sommerfeldt's caption.sty
% which will override IEEEtran.cls' handling of captions and this will result
% in non-IEEE style figure/table captions. To prevent this problem, be sure
% and invoke subfig.sty's "caption=false" package option (available since
% subfig.sty version 1.3, 2005/06/28) as this is will preserve IEEEtran.cls
% handling of captions.
% Note that the Computer Society format requires a larger sans serif font
% than the serif footnote size font used in traditional IEEE formatting
% and thus the need to invoke different subfig.sty package options depending
% on whether compsoc mode has been enabled.
%
% The latest version and documentation of subfig.sty can be obtained at:
% http://www.ctan.org/pkg/subfig




% *** FLOAT PACKAGES ***
%
%\usepackage{fixltx2e}
% fixltx2e, the successor to the earlier fix2col.sty, was written by
% Frank Mittelbach and David Carlisle. This package corrects a few problems
% in the LaTeX2e kernel, the most notable of which is that in current
% LaTeX2e releases, the ordering of single and double column floats is not
% guaranteed to be preserved. Thus, an unpatched LaTeX2e can allow a
% single column figure to be placed prior to an earlier double column
% figure.
% Be aware that LaTeX2e kernels dated 2015 and later have fixltx2e.sty's
% corrections already built into the system in which case a warning will
% be issued if an attempt is made to load fixltx2e.sty as it is no longer
% needed.
% The latest version and documentation can be found at:
% http://www.ctan.org/pkg/fixltx2e


%\usepackage{stfloats}
% stfloats.sty was written by Sigitas Tolusis. This package gives LaTeX2e
% the ability to do double column floats at the bottom of the page as well
% as the top. (e.g., "\begin{figure*}[!b]" is not normally possible in
% LaTeX2e). It also provides a command:
%\fnbelowfloat
% to enable the placement of footnotes below bottom floats (the standard
% LaTeX2e kernel puts them above bottom floats). This is an invasive package
% which rewrites many portions of the LaTeX2e float routines. It may not work
% with other packages that modify the LaTeX2e float routines. The latest
% version and documentation can be obtained at:
% http://www.ctan.org/pkg/stfloats
% Do not use the stfloats baselinefloat ability as the IEEE does not allow
% \baselineskip to stretch. Authors submitting work to the IEEE should note
% that the IEEE rarely uses double column equations and that authors should try
% to avoid such use. Do not be tempted to use the cuted.sty or midfloat.sty
% packages (also by Sigitas Tolusis) as the IEEE does not format its papers in
% such ways.
% Do not attempt to use stfloats with fixltx2e as they are incompatible.
% Instead, use Morten Hogholm'a dblfloatfix which combines the features
% of both fixltx2e and stfloats:
%
% \usepackage{dblfloatfix}
% The latest version can be found at:
% http://www.ctan.org/pkg/dblfloatfix




% *** PDF, URL AND HYPERLINK PACKAGES ***
%
\usepackage{url}
% url.sty was written by Donald Arseneau. It provides better support for
% handling and breaking URLs. url.sty is already installed on most LaTeX
% systems. The latest version and documentation can be obtained at:
% http://www.ctan.org/pkg/url
% Basically, \url{my_url_here}.




% *** Do not adjust lengths that control margins, column widths, etc. ***
% *** Do not use packages that alter fonts (such as pslatex).         ***
% There should be no need to do such things with IEEEtran.cls V1.6 and later.
% (Unless specifically asked to do so by the journal or conference you plan
% to submit to, of course. )


% correct bad hyphenation here
\hyphenation{op-tical net-works semi-conduc-tor}


\begin{document}
%
% paper title
% Titles are generally capitalized except for words such as a, an, and, as,
% at, but, by, for, in, nor, of, on, or, the, to and up, which are usually
% not capitalized unless they are the first or last word of the title.
% Linebreaks \\ can be used within to get better formatting as desired.
% Do not put math or special symbols in the title.
\title{Reinforcement Learning with Deep Q-Networks for Classical Arcade Games}


% author names and affiliations
% use a multiple column layout for up to three different
% affiliations
\author{\IEEEauthorblockN{Gabriel Ewing}
\IEEEauthorblockA{Department of Electrical Engineering\\and Computer Science\\
Case Western Reserve University\\
Cleveland, OH\\
gre5@case.edu}
}

% conference papers do not typically use \thanks and this command
% is locked out in conference mode. If really needed, such as for
% the acknowledgment of grants, issue a \IEEEoverridecommandlockouts
% after \documentclass

% for over three affiliations, or if they all won't fit within the width
% of the page, use this alternative format:
% 
%\author{\IEEEauthorblockN{Michael Shell\IEEEauthorrefmark{1},
%Homer Simpson\IEEEauthorrefmark{2},
%James Kirk\IEEEauthorrefmark{3}, 
%Montgomery Scott\IEEEauthorrefmark{3} and
%Eldon Tyrell\IEEEauthorrefmark{4}}
%\IEEEauthorblockA{\IEEEauthorrefmark{1}School of Electrical and Computer Engineering\\
%Georgia Institute of Technology,
%Atlanta, Georgia 30332--0250\\ Email: see http://www.michaelshell.org/contact.html}
%\IEEEauthorblockA{\IEEEauthorrefmark{2}Twentieth Century Fox, Springfield, USA\\
%Email: homer@thesimpsons.com}
%\IEEEauthorblockA{\IEEEauthorrefmark{3}Starfleet Academy, San Francisco, California 96678-2391\\
%Telephone: (800) 555--1212, Fax: (888) 555--1212}
%\IEEEauthorblockA{\IEEEauthorrefmark{4}Tyrell Inc., 123 Replicant Street, Los Angeles, California 90210--4321}}




% use for special paper notices
%\IEEEspecialpapernotice{(Invited Paper)}




% make the title area
\maketitle

% As a general rule, do not put math, special symbols or citations
% in the abstract
\begin{abstract}
    I implemented deep neural networks for playing Atari 2600 games using Q-Learning.
    We create Voronoi diagrams for two-dimensional maps with point obstacles.
    We explore the algebraic and geometric intricacies of the Sweepline algorithm used
    to build the diagrams. We review the concept of a k-Nearest-Neighbors classifier
    and use the completed Voronoi diagram to find nearest neighbors.
\end{abstract}

% no keywords




% For peer review papers, you can put extra information on the cover
% page as needed:
% \ifCLASSOPTIONpeerreview
% \begin{center} \bfseries EDICS Category: 3-BBND \end{center}
% \fi
%
% For peerreview papers, this IEEEtran command inserts a page break and
% creates the second title. It will be ignored for other modes.
\IEEEpeerreviewmaketitle\



\section{Introduction}
\noindent\ Over the past few decades, computers have been able to play a series
of increasingly complex games at a high level, perhaps most notably in Deep
Blue's defeat of reigning world chess champion Garry Kasparov in 1997 and
AlphaGo's victory against top Go player Lee Sedol in 2016. Complexity, in this
case, is mostly tied to the \emph{branching factor} of the game in question;
that is, the number of actions that are available to a player at a point in
time.  While chess and Go are both complex beyond the point of human
understanding, chess was won much sooner by computers because it has a
significantly lower branching factor than Go.\\
\indent\ Despite their massive possible game trees, both of the aforementioned
games are laughably simple compared to the real world in which we live every
day. This is in part due to the fact that the state at any given time for both
games can be entirely encoded by a relatively small number of bits, whereas the
real world has meaningful state down to the atomic level and a functionally
infinite branching factor.  The low-dimensional game representation means that
vision, one of the most elaborate functions of our brains, has little to no
bearing on our ability to play either game.\\
\indent\ The natural next targets for computational game-playing, then, are
games that require some visual processing to play effectively. In 2013,
Bellemare et al.\ published the Arcade Learning Environment
(ALE)~\cite{bellemare13arcade}, which allows programmatic access to an emulator
for video games on the Atari 2600 console.\\
\indent\ Note that in this paper we mainly follow Mnih et
al.~\cite{mnih2013playing}, and omit some details from that paper for brevity,
but attempt to expand on areas that were given less attention in that paper and
we found interesting during our implementation.

\subsection{Q-Learning}\label{ql}
\noindent\ Parallel to the trend of increasing complexity, focus has shifted
from explicitly teaching computers to play games to having the computers learn
for themselves from experience. This is in part due to stronger general
interest in machine learning, and in part due to the learning approach
surpassing the teaching approach.\\
\indent\ The subfield of machine learning most relevant to gameplay is called
reinforcement learning. Reinforcment learning is the problem of solving Markov
Decision Processes (MDPs).  An MDP is a task consisting of a set of states, a
set of actions, a set of transition probabilities from the set of (state,
action) pairs to the set of outcome states, a set of starting states, and a
reward function. At each step of the MDP, the agent observes a state, selects
an action, and is given a reward signal. The reward signal is what the agent
should try to optimize.\\
\indent\ A prominent classical reinforcement learning approach is Q-Learning.
In Q-Learning, the agent maintains a function that represents the value of a
given (state, action) pair: that is, the expected cumulative reward received by
taking an action in a state. If the agent has learned a good Q-function, it can
then exploit that by taking the action with the maximal Q-value in a given
state.\\
\indent\ Usually, the Q-function is parameterized by a linear combination of
the state features. Its weights are updated by the \emph{Bellman equation}:
\begin{equation}
    Q_{new}(s, a)\mathrel{+}=\alpha\left(R(s,a)+\gamma\max_{a^{'}}Q(s^{'},a^{'})\right)
\end{equation}
$\alpha$ is the \emph{learning rate}, which controls how much the weights move
at each update.  $R(s, a)$ is the reward received after taking action $a$ in
state $s$. $\gamma$ is the \emph{discount factor}, representing how much less
a reward at a future step is worth than a reward at the current step. $s^{'}$
is the state transitioned to after taking action $a$.\\
\indent\ Rather than use a set of linear weights for the Q-function, Deep
Q-Learning uses a represents the Q-function via a neural network. We will
explore this concept further in Section~\ref{deep_q}.

\subsection{Arcade Learning Environment}
\noindent\ The ALE provides an interface to game binaries for Python and Java
(binaries must be acquired from third-party sites). After a game is
initialized, play proceeds in a loop of the agent sending an action, receiving
a reward, and requesting the frame of the current game state.\\
\indent\ Frames are given as 210$\times$160 pixel images with 128 colors. To
reduce the computational load, Mnih et al.~\cite{mnih2013playing} downsampled
the frame to 110$\times$84 and converted the colors to grayscale. We also
adopted this approach. Conveniently, the ALE has an\ API call to get a
grayscale frame instead of RGB.\\
\indent\ The actions available to an agent generally include up, down, left,
right, and game-specific actions such as jump or shoot. These are encoded as
integers by the ALE, so the agent has no semantic knowledge of the actions
before it starts playing the game.

\section{Methods}
\subsection{Deep Q-Learning}\label{deep_q}
\noindent\ As mentioned in Section~\ref{ql}, Deep Q-Learning uses a neural
network to represent the Q-function. This formulation has a couple benefits.
First, deep neural networks are capable of learning extremely complex value
functions, whereas linearly-parameterized functions are limited by the
complexity of the input features. This is important for any game with a large
state space, such as those we are exploring here. Second, deep neural networks
have been the driving force of the recent revolutionary improvements in
computer vision. For games with visually-represented states, we can take
advantage of those improvements.\\
\indent\ There are some differences between the use and updating of a deep
Q-network and a linearly-parameterized Q-function.

\subsubsection{Action Representation}
\noindent\ One possible network architecture would be to have a separate
Q-network for each possible action. We could then feed a state through each
network and select the action with the highest Q-value. This approach would
work, but it ignores some possible performance improvements and recent
discoveries in neural networks.\\
\indent\ Neural networks are computationally expensive to train, and are even
more expensive when the number of weights in the network is high. This means
that when possible, we should try to construct our architecture so that the
number of weights that need to be updated is minimized. One way to do this is
to only use one network for all of the actions, and have each action correspond
to one of the network's output nodes.\\
\indent\ Besides the computational improvements, this takes advantage of the
fact that the early-layer processing for each action should be virtually
identical, since they all require knowledge of the relevant portions of the
game state derived from the visual input.  By using shared early layers, we
ensure that useful visual processing patterns that would have only been
``discovered'' by one network in a separate-action architecture are instead
shared by all of the actions.\\
\indent\ This is similar to the principal that we saw in class where
early-layer weights for image classification tasks can usually be reused
effectively, even for classification on images that would seem unrelated to
those that produced the original weights.

\subsubsection{Network Feedback}
\noindent\ The Bellman equation, discussed in Section~\ref{ql}, is a
\emph{prescriptive} update equation; that is, it gives the steps that should be
taken to update the Q-function. When we have a linear set of weights, the
Bellman equation is sufficient and complete because we know exactly how the
outputs will change with any alteration of the weights.\\
\indent\ In contrast, with a Q-network, we need a \emph{descriptive} update
formulation.  Since the backpropagation equations for neural networks are
established and widely available, we need to \emph{tell the network what it
should produce} and then train it via backpropagation to match that output.\\
\indent\ Mnih et al.~\cite{mnih2013playing} found a way to derive an error
value for backpropagation given a reward and a transition:
\begin{equation}
    Err(r, s, a, s^{'})=r+\gamma\max_{a'}Q(s^{'},a^{'})-Q(s,a)
\end{equation}
This still uses the Bellman contraction from a reward and future Q-value to
figure out how ``wrong'' the current Q-value is.

\subsection{Network Architecture}
\noindent\ We mostly follow the architecture presented
in~\cite{mnih2013playing}, with some minor alterations.

\subsubsection{Downsampling}
\noindent\ As mentioned previously, they downsampled the input frames to
110$\times$84. Furthermore, due to their reliance on a specific GPU convolution
implementation that required square inputs, they truncated the sides of each
downsampled frame so that the resulting output was 84$\times$84. With modern
neural network libraries, this is unnecessary, so we avoid the truncation.
However, to make the modular arithmetic work nicely, we downsample to
108$\times$84 rather than 110$\times$84.

\subsubsection{Frame Stacking}
\noindent\ To reduce the number of network operations per game step, they
stacked four downsampled frames simultaneously and used that as the input to
the network. This means that the network only is called to get a new action
every four game steps, and the agent simply repeats the previous action in
between. We adopt this strategy.

\subsubsection{Convolutional Layers}
\noindent\ There are two convolutional layers in the Q-network: first, a layer
of 16 8$\times$8 convolution patches, followed by a layer of 32 4$\times$4
convolution patches. Each uses TensorFlow's rectified linear unit activation
function (tf.nn.relu).

\subsubsection{Fully-Connected}
\noindent\ Following the convolutional layers is a fully-connected layer with
128 units, also using the rectified linear activation function. The
fully-connected layer is connected to the output layer, which has a number of
nodes corresponding to the number of actions available in the given game.

\subsection{Training}
\subsubsection{Experience Replay}
\noindent\ Classical Q-Learning updates the Q-function with the last transition
seen by the MDP.\ This would probably work in the deep Q-Learning paradigm, but
the authors in~\cite{mnih2013playing} found what is likely an improvement.
Instead of throwing away the transitions after an update is performed, they
store a large number of the most recent transitions. At each update step, they
sample from this memory and update the network using the sampled transitions.
The main benefit of this is that the network does not deviate too much when the
state goes in one direction for an extended period of time.\\
\indent We discovered that each recorded transition requires quite a bit of
data: the prior frame stack, the next frame stack, the action taken, and the
average reward. Just storing the Q-values is insufficient, because the network
update is supposed to be done with the error from the current weights, not a
previous set of weights. Additionally, we have to store the action taken,
because the action that maximizes the Q-value may have changed since the
transition was recorded.

\subsection{Data Structures}
\noindent\ There are two fundamental data structures used in the construction of the Voronoi
diagram.

\subsubsection{Beach Line}
\noindent\ The first data structure represents the beach line. Optimally,
this would be implemented as a balanced binary search tree such as an AVL tree or
red-black tree. In this tree, leaf nodes represent obstacles and
the internal nodes represent the intersections between arcs~\cite{mitslides}.\\
\indent In the interest of ease of implemenation, we instead use a
doubly-linked list to store the beach line. Nodes in the list contain
obstacles, and two adjacent nodes have pointers to each other and to the
growing Voronoi edge between them. Using a list instead of a search tree
increases the complexity of locating the arc that a new obstacle falls under
from $O(\log n)$ to $O(n)$.\\
\indent In this setup, a node has two adjacent edges, each with an intersection
at its endpoint. When a node is first encountered by the sweepline, its arc is a
vertical line, and so only intersects the beach line once. However, the equation
for intersections in Section~\ref{obstacle_events} returns two copies of this
intersection. This allows us to create the arcs without having to handle an
additional case.\\
\indent An obstacle can have multiple nodes associated with it. Each time the
obstacle's parabola is split by an obstacle event, the single node containing
that segment of the arc splits into two. This corresponds to the obstacle's
Voronoi region gaining an incident edge.\\
\indent When a circle event occurs, the corresponding obstacle's node is removed
from the list and its neighbors are joined together, forming an intersection on the
Voronoi diagram and a new edge leading out of it.

\subsubsection{Event Queue}
\noindent\ The second data structure is a priority queue containing the events, which we refer to as
the \emph{event queue}. We implement this using a binary heap, Python's \emph{heapq}.
Initially, the event queue contains all of the obstacles minus duplicate obstacles,
beginning with the obstacle with maximal y-value. Ordering ties are broken by
placing the event with the lower x-value first.

\subsubsection{Interactions}
\noindent\ There is some overlap between the responsibilities of the beach line and the event
queue. Both of them keep track of circle events: each node in the beach line
has an associated circle event, if any, and the event queue contains the
active circle events in addition to the obstacle events. After each event,
the nearby nodes in the beach line are examined for circle event updates. This
includes both adding potential new circle events if a node was added to the
beach line, and removing circle events if a new obstacle falls inside the
circle. If a circle event is removed from the beach line, it also is removed
from the event queue. This is a costly operation because lookup in a heap
costs $O(n)$.

\subsection{Nearest Neighbors}\label{knn}
\noindent\ Given a complete and robust Voronoi diagram, KNN classification can be done
in a similar manner to a breadth-first search. To classify a test point, the
first step is to locate the Voronoi region that it resides in. The obstacle
corresponding to that region is the nearest neighbor. If $k>1$, the obstacles
in regions adjacent to the nearest neighbor are added to a list. The second
nearest neighbor will be the closest obstacle in the list. Additional neighbors
can be added in a similar manner.\\
\indent Classification in this manner requires that edges have knowledge of
the obstacles on both sides of them, which our Voronoi implementation does
not. We output a list of edges and a dictionary mapping each obstacle to
the edges surrounding its region.\\
\indent We modify the nearest-neighbor implementation
as follows: after the nearest neighbor is found, it is removed from the list
of obstacles and the Voronoi diagram is re-calculated. The second nearest
neighbor is now the obstacle for the region that the test point resides in.
This can be repeated for the $k$ nearest neighbors. This impacts
the computational complexity of the classification, as creating $k$ Voronoi
diagrams increases the runtime by a factor of $k$.\\
\indent The standard Voronoi diagram has at its perimeter a number of edges
that extend infinitely in the plane. It is difficult to efficiently calculate
whether a point lies in one of these semi-bounded regions. Because of this,
we take a post-processing step for the Voronoi diagram that imposes a bounding
box around the obstacles; all the obstacles and intersections lie within
the box, and any test points for classification must as well.\\
\indent An additional post-processing step is to sort the Voronoi regions
by the minimum value of $y$ of any point in the region. This allows us to
do a binary search to find the region with the highest $y$ that is not
entirely higher than the test point. From that point, we proceed linearly
through the regions in order of decreasing highest $y$, checking whether
the test point belongs in that region, until the correct region is found.
This heuristic improves the runtime
of the search for the region that the point falls into once we have the
diagram, as all regions with higher minimum $y$ can be ignored. The
sorting step itself negates at least some of the runtime improvements, but
in a better classification scheme (search within a diagram rather than
re-calculating the diagram) it would likely be worth the one-time costs.

\section{Results}
\subsection{Voronoi Diagrams}
\noindent\ shows the results of creating a Voronoi diagram with fifty
point obstacles with $x$ and $y$ coordinates randomly selected from a uniform
distribution between $0$ and $10$. While we did not analytically verify the
results, the output looks intuitively correct. Unfortunately, there is a latent
bug in our implementation that causes an error with probability proportional
to the number of obstacles, so this is close to the upper limit of the problem
size that we can currently generate results for.

\subsection{Nearest Neighbors}
\noindent\ As a result of the aforementioned issues, testing the nearest-neighbors
classification on any real dataset is not feasible; training generally requires
significantly more than fifty input points. However, we are able to produce the
$k$ nearest neighbors to an input point, for values of $k$ up to about $10$.
This means that the theoretical challenges are mostly handled. Creating a more
robust classifier would be mostly a matter of scaling and some amount of
debugging. As mentioned in Section~\ref{knn}, the algorithm that we used
to find the neighbors was fairly inefficient, but on such a small dataset this
did not impose any noticeable runtime degradation.

\section{Discussion}
\noindent\ Our results are generally correct, but we wish that we could have
implemented the most efficient algorithms and data structures for the diagram
construction and KNN classification. During construction, we would prefer to
have used a balanced binary tree rather than a doubly-linked list for the beach
line. In the classification step, theoretical complexity improvements would have
to be taken from modified data structures created during the diagram construction.
If the edges list had included the vertices at either end of the edge, we could
have made improvements in the time to find the region that a point belongs to.
If the edges list had included the obstacles for the regions incident to the
edge, we could have done an exhaustive search starting from the nearest neighbor
rather than re-creating the Voronoi diagram. These improvements are theoretically
simple to implement but we did not consider them during implementation of the
Sweepline algorithm and lacked sufficient time to modify it at the end.

% An example of a floating figure using the graphicx package.
% Note that \label must occur AFTER (or within) \caption.
% For figures, \caption should occur after the \includegraphics.
% Note that IEEEtran v1.7 and later has special internal code that
% is designed to preserve the operation of \label within \caption
% even when the captionsoff option is in effect. However, because
% of issues like this, it may be the safest practice to put all your
% \label just after \caption rather than within \caption{}.
%
% Reminder: the "draftcls" or "draftclsnofoot", not "draft", class
% option should be used if it is desired that the figures are to be
% displayed while in draft mode.
%
%\begin{figure}[!t]
%\centering
%\includegraphics[width=2.5in]{myfigure}
% where an .eps filename suffix will be assumed under latex, 
% and a .pdf suffix will be assumed for pdflatex; or what has been declared
% via \DeclareGraphicsExtensions.
%\caption{Simulation results for the network.}
%\label{fig_sim}
%\end{figure}

% Note that the IEEE typically puts floats only at the top, even when this
% results in a large percentage of a column being occupied by floats.


% An example of a double column floating figure using two subfigures.
% (The subfig.sty package must be loaded for this to work.)
% The subfigure \label commands are set within each subfloat command,
% and the \label for the overall figure must come after \caption.
% \hfil is used as a separator to get equal spacing.
% Watch out that the combined width of all the subfigures on a 
% line do not exceed the text width or a line break will occur.
%
%\begin{figure*}[!t]
%\centering
%\subfloat[Case I]{\includegraphics[width=2.5in]{box}%
%\label{fig_first_case}}
%\hfil
%\subfloat[Case II]{\includegraphics[width=2.5in]{box}%
%\label{fig_second_case}}
%\caption{Simulation results for the network.}
%\label{fig_sim}
%\end{figure*}
%
% Note that often IEEE papers with subfigures do not employ subfigure
% captions (using the optional argument to \subfloat[]), but instead will
% reference/describe all of them (a), (b), etc., within the main caption.
% Be aware that for subfig.sty to generate the (a), (b), etc., subfigure
% labels, the optional argument to \subfloat must be present. If a
% subcaption is not desired, just leave its contents blank,
% e.g., \subfloat[].


% An example of a floating table. Note that, for IEEE style tables, the
% \caption command should come BEFORE the table and, given that table
% captions serve much like titles, are usually capitalized except for words
% such as a, an, and, as, at, but, by, for, in, nor, of, on, or, the, to
% and up, which are usually not capitalized unless they are the first or
% last word of the caption. Table text will default to \footnotesize as
% the IEEE normally uses this smaller font for tables.
% The \label must come after \caption as always.
%
%\begin{table}[!t]
%% increase table row spacing, adjust to taste
%\renewcommand{\arraystretch}{1.3}
% if using array.sty, it might be a good idea to tweak the value of
% \extrarowheight as needed to properly center the text within the cells
%\caption{An Example of a Table}
%\label{table_example}
%\centering
%% Some packages, such as MDW tools, offer better commands for making tables
%% than the plain LaTeX2e tabular which is used here.
%\begin{tabular}{|c||c|}
%\hline
%One & Two\\
%\hline
%Three & Four\\
%\hline
%\end{tabular}
%\end{table}


% Note that the IEEE does not put floats in the very first column
% - or typically anywhere on the first page for that matter. Also,
% in-text middle ("here") positioning is typically not used, but it
% is allowed and encouraged for Computer Society conferences (but
% not Computer Society journals). Most IEEE journals/conferences use
% top floats exclusively. 
% Note that, LaTeX2e, unlike IEEE journals/conferences, places
% footnotes above bottom floats. This can be corrected via the
% \fnbelowfloat command of the stfloats package.


\section{Conclusion}
We successfully implemented a Voronoi diagram algorithm for point
obstacles in two dimensions. While there exist easier Voronoi algorithms
to implement, the Sweepline algorithm was interesting geometrically
and promises good performance results. We used the completed Voronoi
diagram to create a k-Nearest-Neighbors classifier that runs correctly
on toy datasets.


% conference papers do not normally have an appendix





% trigger a \newpage just before the given reference
% number - used to balance the columns on the last page
% adjust value as needed - may need to be readjusted if
% the document is modified later
%\IEEEtriggeratref{8}
% The "triggered" command can be changed if desired:
%\IEEEtriggercmd{\enlargethispage{-5in}}

% references section

% can use a bibliography generated by BibTeX as a .bbl file
% BibTeX documentation can be easily obtained at:
% http://mirror.ctan.org/biblio/bibtex/contrib/doc/
% The IEEEtran BibTeX style support page is at:
% http://www.michaelshell.org/tex/ieeetran/bibtex/
\nocite{*}
\bibliographystyle{IEEEtran}
% argument is your BibTeX string definitions and bibliography database(s)
\bibliography{references.bib}
%
% <OR> manually copy in the resultant .bbl file
% set second argument of \begin to the number of references
% (used to reserve space for the reference number labels box)
% \begin{thebibliography}{1}

% \end{thebibliography}




% that's all folks
\end{document}


